\section*{Описание реализуемого класса и методов}

Для выполнения задания будут реализованы два класса.

\verb|BstNode| будет являться узлом дерева.
Каждый узел дерева является объектом класса \verb|BstNode|.
\verb|BstNode| содержит в себе данные и ссылки на левого и правого потомков.

Основной класс \verb|BST| предназначен для работы с деревом.
В нём будут реализованы все методы, требуемые в задании.
Класс \verb|BST| содержит только ссылку на корневой узел дерева.

Для выполнения задания используется язык \verb|Python3| версии \verb|3.9|

Дополнительно будут реализованы следующие методы:

\verb|def get(self, value, default=None) -> Optional[BstNode]:| ---
Метод для получения узла дерева по значению.

\verb|def get_parent(self, value, default=None) -> Optional[BstNode]:| ---
Метод для получения родительского узла к узлу с переданным значением.

\verb|def _height(self, node) -> int:| ---
Метод для получения высоты дерева.

\verb|def get_str(self, node: BstNode = _MISSING, indent=0, arrow='--->')|
\verb|-> Generator:| ---
Метод для получения строки, отображающей дерево.

\verb|def __check_type(value: Any, t: Type = int) -> None:| ---
Метод для проверки \verb|value| на соответствие типу \verb|t|.

\verb|def get_max_node(node) -> Optional[BstNode]:| ---
Метод для получения узла с максимальным значением в дереве с корнем в \verb|node|.